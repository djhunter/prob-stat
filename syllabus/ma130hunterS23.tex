% Options for packages loaded elsewhere
\PassOptionsToPackage{unicode}{hyperref}
\PassOptionsToPackage{hyphens}{url}
%
\documentclass[
  twoside]{article}
\usepackage{amsmath,amssymb}
\usepackage{lmodern}
\usepackage{iftex}
\ifPDFTeX
  \usepackage[T1]{fontenc}
  \usepackage[utf8]{inputenc}
  \usepackage{textcomp} % provide euro and other symbols
\else % if luatex or xetex
  \usepackage{unicode-math}
  \defaultfontfeatures{Scale=MatchLowercase}
  \defaultfontfeatures[\rmfamily]{Ligatures=TeX,Scale=1}
\fi
% Use upquote if available, for straight quotes in verbatim environments
\IfFileExists{upquote.sty}{\usepackage{upquote}}{}
\IfFileExists{microtype.sty}{% use microtype if available
  \usepackage[]{microtype}
  \UseMicrotypeSet[protrusion]{basicmath} % disable protrusion for tt fonts
}{}
\makeatletter
\@ifundefined{KOMAClassName}{% if non-KOMA class
  \IfFileExists{parskip.sty}{%
    \usepackage{parskip}
  }{% else
    \setlength{\parindent}{0pt}
    \setlength{\parskip}{6pt plus 2pt minus 1pt}}
}{% if KOMA class
  \KOMAoptions{parskip=half}}
\makeatother
\usepackage{xcolor}
\usepackage[left=1.0in, right=1.0in, top=0.8in, bottom=0.4in]{geometry}
\usepackage{graphicx}
\makeatletter
\def\maxwidth{\ifdim\Gin@nat@width>\linewidth\linewidth\else\Gin@nat@width\fi}
\def\maxheight{\ifdim\Gin@nat@height>\textheight\textheight\else\Gin@nat@height\fi}
\makeatother
% Scale images if necessary, so that they will not overflow the page
% margins by default, and it is still possible to overwrite the defaults
% using explicit options in \includegraphics[width, height, ...]{}
\setkeys{Gin}{width=\maxwidth,height=\maxheight,keepaspectratio}
% Set default figure placement to htbp
\makeatletter
\def\fps@figure{htbp}
\makeatother
\setlength{\emergencystretch}{3em} % prevent overfull lines
\providecommand{\tightlist}{%
  \setlength{\itemsep}{0pt}\setlength{\parskip}{0pt}}
\setcounter{secnumdepth}{-\maxdimen} % remove section numbering
\usepackage[charter]{mathdesign}
\usepackage{fancyhdr}
\pagestyle{fancy}
\lhead{MA-130, Hunter, Spring 2023}
\rhead{Syllabus, Page \thepage}
\cfoot{}
\setlength{\headsep}{0.2in}

\usepackage{enumitem}

\setenumerate{leftmargin=*}

\newcommand{\vect}[1]{\mathbf{#1}} % vectors in bold face

%\newcommand{\vect}[1]{\mathbf{#1}\,} % vectors in bold face (need thin
                                     % space after)
%\newcommand{\vect}[1]{\vec{#1}} %vectors as arrows

\providecommand{\norm}[1]{\left\lvert#1\right\rvert} % norms as single lines
%\providecommand{\norm}[1]{\left\lVert#1\right\rVert} % norms as double lines

% bold or blackboard bold!?
\newcommand{\NN}{\mathbb{N}}
%\newcommand{\RR}{\mathbf{R}}
\newcommand{\RR}{\mathbb{R}}

% some useful abbreviations
\newcommand{\vx}{\vect{x}}
\newcommand{\vy}{\vect{y}}
\newcommand{\vz}{\vect{z}}
\newcommand{\vu}{\vect{u}}
\newcommand{\vv}{\vect{v}}
\newcommand{\vw}{\vect{w}}
\newcommand{\va}{\vect{a}}
\newcommand{\vb}{\vect{b}}
\newcommand{\vc}{\vect{c}}
\newcommand{\ve}{\vect{e}}
\newcommand{\vf}{\vect{f}}
\newcommand{\vF}{\vect{F}}
\newcommand{\vg}{\vect{g}}
\newcommand{\vh}{\vect{h}}
\newcommand{\vl}{\vect{l}}
\newcommand{\vm}{\vect{m}}
\newcommand{\vn}{\vect{n}}
\newcommand{\vp}{\vect{p}}
\newcommand{\vr}{\vect{r}}
\newcommand{\vs}{\vect{s}}
\newcommand{\vi}{\vect{i}}
\newcommand{\vj}{\vect{j}}
\newcommand{\vk}{\vect{k}}
\newcommand{\vzero}{\vect{0}}
% lower-case greek letters are handled differently: poor man's bold macro
%\newcommand{\vphi}{\pmb{\phi}}

\newcommand{\malename}{Westley} % recurring male name
\newcommand{\femalename}{Buttercup} % recurring female name
\usepackage{booktabs}
\usepackage{longtable}
\usepackage{array}
\usepackage{multirow}
\usepackage{wrapfig}
\usepackage{float}
\usepackage{colortbl}
\usepackage{pdflscape}
\usepackage{tabu}
\usepackage{threeparttable}
\usepackage{threeparttablex}
\usepackage[normalem]{ulem}
\usepackage{makecell}
\usepackage{xcolor}
\ifLuaTeX
  \usepackage{selnolig}  % disable illegal ligatures
\fi
\IfFileExists{bookmark.sty}{\usepackage{bookmark}}{\usepackage{hyperref}}
\IfFileExists{xurl.sty}{\usepackage{xurl}}{} % add URL line breaks if available
\urlstyle{same} % disable monospaced font for URLs
\hypersetup{
  hidelinks,
  pdfcreator={LaTeX via pandoc}}

\author{}
\date{\vspace{-2.5em}}

\begin{document}

\hypertarget{probability-and-statistics-ma-130-westmont-college-spring-2023}{%
\section{Probability and Statistics (MA-130) Westmont College, Spring
2023}\label{probability-and-statistics-ma-130-westmont-college-spring-2023}}

\hypertarget{what-is-this-course-about}{%
\subsection{What is this course
about?}\label{what-is-this-course-about}}

Statistics is the science of making inferences from data. Probability is
the mathematical theory at the foundation of statistical methods. Topics
in probability include random variables, conditional probability,
density functions, discrete and continuous distributions, important
distributions (e.g., normal, binomial, Poisson), the law of large
numbers, and the central limit theorem. Statistical applications include
hypothesis testing, confidence intervals, regression models, and
statistical learning. This course will be a mix of theory and
applications. (Prerequisite: Calculus II.)

\hypertarget{is-there-a-textbook}{%
\subsection{Is there a textbook?}\label{is-there-a-textbook}}

Yes! In fact there are three textbooks, and they are all freely
available online. We will cover a selection of material from the
following sources.

\begin{itemize}
\tightlist
\item
  {[}GS{]}:
  \href{https://math.dartmouth.edu/~prob/prob/prob.pdf}{\emph{Grinstead and Snell's Introduction to Probability}},
  The Chance Project.
\item
  {[}JWHT{]}:
  \href{https://hastie.su.domains/ISLR2/ISLRv2_website.pdf}{\emph{An Introduction to Statistical Learning}},
  James, Witten, Hastie, Tibshirani.
\item
  {[}ER{]}:
  \href{https://www.utstat.toronto.edu/mikevans/jeffrosenthal/book.pdf}{\emph{Probability and Statistics}},
  Evans and Rosenthal.
\end{itemize}

The first unit on probability theory will draw mainly from {[}GS{]},
while the second unit on statistical practice will draw mainly from
{[}JWHT{]}. Occasionally, {[}ER{]} will be consulted as an additional
resource. You are also expected to install R and RStudio on your laptop.

\hypertarget{what-is-the-coursework-and-how-is-it-graded}{%
\subsection{What is the coursework and how is it
graded?}\label{what-is-the-coursework-and-how-is-it-graded}}

You should expect an \textbf{assignment} due the night before every
class meeting. You will submit them on Canvas in a variety of formats
(e.g, scanned PDFs, R scripts, knitted HTML). There will be two written
midterm \textbf{exams}, and a cumulative \textbf{final exam} on the
scheduled date. The following table shows how these assessments are
weighted to determine your final grade.

\begin{tabular}[t]{ll}
\toprule
Assignments & 40\%\\
Midterms & 20\% each\\
Final Exam & 20\%\\
\bottomrule
\end{tabular}

\hypertarget{what-other-policies-should-students-be-aware-of}{%
\subsection{What other policies should students be aware
of?}\label{what-other-policies-should-students-be-aware-of}}

If you miss a significant number of classes, you will almost definitely
do poorly in this class. If you miss more than four classes without a
valid excuse, I reserve the right to terminate you from the course with
a failing grade. Work missed (including tests) without a valid excuse
will receive a zero.

I expect you to check your email on a regular basis. If you use a
non-Westmont email account, please forward your Westmont email to your
preferred account. I'll send out notices on Canvas, so make sure you
receive Canvas notifications in your email.

Learning communities function best when students have academic
integrity. Cheating is primarily an offense against your classmates
because it undermines our learning community. Therefore, dishonesty of
any kind may result in loss of credit for the work involved and the
filing of a report with the Provost's Office. Major or repeated
infractions may result in dismissal from the course with a failing
grade. Be familiar with the College's plagiarism policy, found at
\url{https://www.westmont.edu/office-provost/academic-program/academic-integrity-policy}.

In particular, providing someone with an electronic copy of your work is
a breach of the academic integrity policy. Do not email, post online, or
otherwise disseminate any of the work that you do in this class. If you
keep your work on a repository, make sure it is private. You may work
with others on the assignments, but make sure that you write or type up
your own answers yourself. You are on your honor that the work you hand
in represents your own understanding.

\clearpage

\hypertarget{other-information}{%
\subsection{Other Information}\label{other-information}}

\begin{description} 

\item[Professor:] David J. Hunter, Ph.D.
  (\verb!dhunter@westmont.edu!). Student hours are from 2:00--4:30pm on Tuesdays and Thursdays in Winter Hall 303.

 \item[Tentative Schedule:] The following schedule is a rough first approximation of the topics in [GS], [JWHT], and [ER] that we plan to cover; it is subject to revision at the instructor's discretion. 
  \begin{itemize}
      \item Probability Theory: Random variables, Monte Carlo simulations, discrete and continuous distributions and densities, important distributions, expected value and variance, the law of large numbers, and the central limit theorem.
   \begin{quote}
    \textit{Midterm \#1}     
   \end{quote}
      \item Statistical Practice: Hypothesis testing and confidence intervals, statistical learning models, the bias-variance trade-off, linear regression, logistic regression, cross validation, bootstrapping, model selection. 
   \begin{quote}
    \textit{Midterm \#2}  
   \end{quote}
      \item Further Applications: Time permitting, a selection of topics from principal components analysis, decision trees, random forests, support vector machines, or other statistical learning methods.
   \begin{quote}
    \textit{Final Exam}     (cumulative)
   \end{quote}
  \end{itemize}

\item[Accommodations for Students with Disabilities:] Students who have been diagnosed with a disability (learning, physical or psychological) are strongly encouraged to contact the Disability Services office as early as possible to discuss appropriate accommodations for this course. Formal accommodations will only be granted for students whose disabilities have been verified by the Disability Services office.  These accommodations may be necessary to ensure your equal access to this course.  Please contact the Office of Disability Services (\href{mailto:ods@westmont.edu}{\tt ods@westmont.edu}) or visit \url{https://www.westmont.edu/disability-services} for more information.

\item[Program and Institutional Learning Outcomes:] The
         mathematics department at Westmont College has formulated the
         following learning outcomes for all of its classes. (PLO's)
\begin{enumerate}[noitemsep]
\item Core Knowledge: Students will demonstrate knowledge of the
                  main concepts, skills, and facts of the discipline of
                  mathematics.
\item Communication: Students will be able to communicate mathematical ideas
     following the standard conventions of writing or speaking in the
     discipline.
\item Creativity: Students will demonstrate the ability to formulate and make
     progress toward solving non-routine problems.
\item Christian Connection: Students will incorporate their mathematical skills
     and knowledge into their thinking about their vocations as followers of
     Christ.
         \end{enumerate}
         In addition, the faculty of Westmont College have established common
         learning outcomes for all courses at the institution
         (ILO's). These outcomes are summarized as follows:
(1) Christian Understanding, Practices, and Affections,
(2) Global Awareness and Diversity,
(3) Critical Thinking,
(4) Quantitative Literacy,
(5) Written Communication,
(6) Oral Communication, and
(7) Information Literacy.

\item[Course Learning Outcomes:] The above outcomes are reflected in the
     particular learning outcomes for this course.
     After taking this course, you should be able
     to:
    \begin{itemize}
        \item Demonstrate understanding of the theoretical basis for statistical practice.
             (PLO 1, ILOs 3,4)
        \item Write and evaluate mathematical arguments according to the
             standards of the discipline. (PLO 2,
              ILOs 3,5)
        \item Construct solutions to novel problems,
               demonstrating perseverance in the face of open-ended or
               partially-defined contexts. (PLO 3, ILO 3)
        \item Consider the ethical implications of the subject matter. (PLO 4, ILO 1)
    \end{itemize}
These outcomes will be assessed by assignments and written exams, as described above.

\end{description}

\end{document}
